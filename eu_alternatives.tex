\section{EU-Alternatives}
To strenghten digital Sovereignty, it is imperative that services gets migrated to EU friendly
alternatives. To do this, the requirements can be broken down into different functional areas.
Where each area is corresponds to a dependency. 

\begin{itemize}
    \item Identity \& Access Management - The focus here is employeee authentication and identity
    MFA, employee and customer identity management with AD features 
    \item Cloud compute \& Hosting - To replace services currently held by Azure, 
    which includes API hosting, data ingestion, functions and kubernetes like features
    \item Data Storage \& processing - Azure blob and CosmosDB, currently handle critical functions such as
    HL7v2 conversion, DICOM data uploads. Alternatives should have physical address in the EU, with strong encryption
    \item Development \& DevOps - This should include services with the ability to host source, CI/CD pipeline, 
    issue tracking, version control and support GIT based workflow.
    \item Collaboration \& Productivity tools - To replace service commonly provided by Microsoft. Services here
    are email, documents, internal communication, file storage and file sharing.
    \item Security \& Monitoring - A replacement for Cisco products, 
    should be able to prove services such as VPN, SIEM, logging and zero-trust capabilites.
    \item Patient/customer identity - A replacement for Azure B2C, which handles App user 
    accounts and authentication. Must be able to be integrated into applications.
    \item Legacy On-prem Replacement \& Hybrid Integration - Current setup implements physical SQL servers, legacy SMB shares 
    and physical AD. The AD should be integrated with the IAM service, and legacy servers should be 
    integrated into a single physical setup with a cloud backup, along with tightened physical security access.
\end{itemize}


With all the functional areas, which LHT relies on, outlined the next step is
identify and pick european service providers, who can satisfy the different requirements.
In order to minimize risk and impact, solutions who implement multiple services into
a single provider, such a Azure does, is preferable, however this also brings 
some of the vendor lock-in back into the equation.\\

European alternatives are listed below, 

\begin{itemize}
    \item Identity \& Access Management - Univention, a German based IAM solution, 
    that can handle AD, and their IAM service NUBUS.  
    \item Cloud compute \& Hosting - Scaleway, a French based cloud compute company, which
    includes virtual machines and kubernetes abilities. With data residency inside european borders.
    \item Data Storage \& processing - This requirement can also be delivered by Scaleway, as they
    also provide object storage of large amounts of data aswell as block storage.  
    \item Development \& DevOps - general CI/CD operations can be hosted in a GIT server,
    hosted by Scaleway.
    \item Collaboration \& Productivity tools - LibreOffice is an latvian alternative to the
    Microsoft office suite. As for collaboration tools, NextCloud, which originates from Germany.
    \item Security \& Monitoring - WithSecure from Finland, An XDR solution is chosen, because we want
    more than just an antivirus at enterprise level. Along with LogPoint, a Danish company, which is a 
    SIEM solution, which is an information correlation service.
    \item Patient/Customer identity - Univention's NUBUS has an integration with KeyCloak, which is 
    open-source, that supports OpenID and OAuth2. Which can be self hosted. 
    \item Legacy On-prem Replacement \& Hybrid Integration - The hardware server offings in europe,
    is still in its infancy, a japanese alternative from Fujitsu could be considered.
\end{itemize}

These proposed solutions and how they stack up against the current implementation will be discussed in the next sectiton.

%univention
%scaleway
%assume breach, verify explicitly 

% HUSK CITES
