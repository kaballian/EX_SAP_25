\section{EU-Alternatives}
To strenghten digital Sovereignty, it is imperative that services gets migrated to EU friendly
alternatives. To do this, the requirements can be broken down into different functional areas.
Where each area is corresponds to a dependency. 

\begin{itemize}
    \item Identity \& Accessmanagement - The focus here is employeee authentication and identity
    MFA, employee and customer identity management with AD features 
    \item Cloud compute \& Hosting - To replace services currently held by Azure, 
    which includes API hosting, data ingestion, functions and kubernetes like features
    \item Data Storage \& processing - Azure blob and CosmosDB, currently handle critical functions such as
    HL7v2 conversion, DICOM data uploads. Alternatives should have physical address in the EU, with strong encryption
    \item Development \& DevOps - This should include services with the ability to host source, CI/CD pipeline, 
    issue tracking, version control and support GIT based workflow.
    \item Collaboration \& Productivity tools - To replace service commonly provided by Microsoft. Services here
    are email, documents, internal communication, file storage and file sharing.
    \item Security \& Monitoring - A replacement for Cisco products, 
    should be able to prove services such as VPN, SIEM, logging and zero-trust capabilites.
    \item Patient/customer identity \& - A replacement for Azure B2C, which handles App user 
    accounts and authentication. Must be able to be integrated into applications.
    \item Legacy On-prem Replacement \& Hybrid Integration - Current setup implements physical SQL servers, legacy SMB shares 
    and physical AD. The AD should be integrated with the IAM service, and legacy servers should be 
    integrated into a single physical setup with a cloud backup, along with tightened physical security access.
\end{itemize}


With all the functional areas, which LHT relies on, outlined, the next step is
identify and pick european service providers, who can satisfy the different requirements.
In order to minimize risk and impact, solutions who implement multiple services into
a single provider, such a Azure does, is preferable, however this also brings 
some of the vendor lock-in back into the equation.\\

European alternatives are listed below, 

\begin{itemize}
    \item Identity \& Accessmanagement -  
    \item Cloud compute \& Hosting - 
    \item Data Storage \& processing - 
    \item Development \& DevOps - 
    \item Collaboration \& Productivity tools - 
    \item Security \& Monitoring - 
    \item Patient/customer identity \& - 
    \item Legacy On-prem Replacement \& Hybrid Integration - 
\end{itemize}


%assume breach, verify explicitly 
