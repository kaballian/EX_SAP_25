\section{Risk Assessment}
To anayse how the digital soverignty is impacted LHT's inherited IT-environment, 
a risk assessment is carried out. It is done by using a impact-likelyhood matrix, seen on
figure: \ref{fig:risk_v_impact}. This model is comprised of nine sectons, each of a 
severity level, which ranges from low to critial. This is based on the likehood of occurence
and the impact on the company, whether it be operational, legal or stategic. 

The axis of the model are impact and likehood, 


%jeg ved ikke om det er lovligt at bruge et billede uden at sige hvor jeg har det fra??
\begin{figure}[!htb]
    \centering
    \includegraphics[width=0.8\textwidth]{risk_matrix.png}
    \caption{risk vs impact matrix, this is used to visually position LHT's risks}
    \label{fig:risk_v_impact}
\end{figure}


From this model it is possible to construct a table of the risks, pertaining to the 
current providers and dependency severity, found in table: \ref{tab:dependency} following
the Methodology: $risk = Likelihood \cdot Impact $. The basis for this Methodology can be
found at OWASP \cite{OWASP}




\begin{table}[htb]
\centering
\renewcommand{\arraystretch}{1.2}
\begin{tabular}{|l|l|l|l|l|}
\hline
\textbf{Provider / Service} &\textbf{Likelihood} &\textbf{Impact} &
\textbf{Severity} &\textbf{Justification}\\
\hline

Microsoft Azure &  &    &  &           \\\hline
Azure AD / B2C  &  &    &  &           \\\hline
Microsoft 365   &  &    &  &           \\\hline
GitHub          &  &    &  &           \\\hline
Cisco ASA / VPN &  &    &  &         \\\hline
Cosmos DB       &  &    &  &           \\\hline
Azure Functions &  &    &  &           \\\hline
Slack           &  &    &  &       \\\hline
Google Cloud    &  &    &  &         \\\hline
On-prem AD      &  &    &  &            \\\hline
Mediscan tools  &  &    &  &          \\\hline
AnyConnect VPN  &  &    &  &         \\
\hline
\end{tabular}
\caption{Summary of current providers and sovereignty-related dependency severity.}
\label{tab:risk table}
\end{table}



%brug en blød risk/impact matrix til den generelle fremgang,
%herefter link til https://owasp.org/www-community/OWASP_Risk_Rating_Methodology
%og giv et eksempel på hvordan det kunne se ud der fra, ikke nødvendigt at bruge billeder
%nævn at fokuset på rapporten er ikke risk management, men digital sovereignty



%nævn det med cisco ASA FIREPOWER VULNERABILITY

%nævn AD forest og at den er federated

%nævn manufacturig systems med static IP 