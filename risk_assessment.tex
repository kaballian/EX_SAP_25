\section{Risk Assessment}
To anayse how the digital soverignty is impacted LHT's inherited IT-environment, 
a risk assessment is carried out. It is done by using a impact-likelyhood matrix, seen on
figure: \ref{fig:risk_v_impact}. This model is comprised of nine sectons, each of a 
severity level, which ranges from low to critial. This is based on the likehood of occurence
and the impact on the company, whether it be operational, legal or stategic. 


\begin{figure}[!htb]
    \centering
    \includegraphics[width=0.8\textwidth]{risk_matrix.png}
    \caption{risk vs impact matrix}
    \label{fig:risk_v_impact}
\end{figure}


%brug en blød risk/impact matrix til den generelle fremgang,
%herefter link til https://owasp.org/www-community/OWASP_Risk_Rating_Methodology
%og giv et eksempel på hvordan det kunne se ud der fra, ikke nødvendigt at bruge billeder
%nævn at fokuset på rapporten er ikke risk management, men digital sovereignty



%nævn det med cisco ASA FIREPOWER VULNERABILITY

%nævn AD forest og at den er federated

%nævn manufacturig systems med static IP 