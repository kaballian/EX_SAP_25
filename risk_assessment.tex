\section{Risk Assessment}
To anayse how the digital soverignty is impacted LHT's inherited IT-environment, 
a risk assessment is carried out. It is done by using a impact-likelyhood matrix, seen on
figure: \ref{fig:risk_v_impact}. This model is comprised of nine sectons, each of a 
severity level, which ranges from low to critial. This is based on the likehood of occurence
and the impact on the company, whether it be operational, legal or stategic. 

The axis of the model are impact and likehood, where every step out of the axis 
increases in the steps, low, medium and high


%jeg ved ikke om det er lovligt at bruge et billede uden at sige hvor jeg har det fra??
\begin{figure}[!htb]
    \centering
    \includegraphics[width=0.8\textwidth]{risk_matrix.png}
    \caption{risk vs impact matrix, this is used to visually position LHT's risks. The 
    The product of this matrix is indicated as \textit{severity} in table \ref{tab:risk table}, 
    picture from AlertMedia: \cite{RISKMATRIX}}
    \label{fig:risk_v_impact}
\end{figure}


From this model it is possible to construct a table of the risks, pertaining to the 
current providers and dependency severity, found in table: \ref{tab:dependency}, following
the Methodology: $risk = Likelihood \cdot Impact $. The basis for this Methodology can be
found at OWASP\cite{OWASP}. However in this case likelyhood is not only the question, 
\textit{"will the service break?"}, but also how propable it is, that any impact on this 
dependency, will bring negative consequences for LHT as a whole?.


%brug lige simon til at udfylde den her tabel, jeg aner ikke hvad jeg laver
\begin{table}[!htb]
\centering
\renewcommand{\arraystretch}{1.2}

\begin{tabular}{|l|l|l|l|l|}
\hline
\textbf{Provider / Service} &\textbf{Likelihood} &\textbf{Impact} &
\textbf{Severity} &\textbf{Justification}\\
\hline

Microsoft Azure &  LOW & HIGH  & MEDIUM & Given the size of Microsoft. Constant resources are assigned to maintain Azure.\\\hline
Azure AD / B2C  &  LOW & HIGH & MEDIUM & Same justification as Microsoft Azure.          \\\hline
Microsoft 365   &  MEDIUM & LOW & LOW & 365 includes the office suite, thus they OutLook, where a lot of attack comes from phising.\\\hline
GitHub          &  MEDIUM & MEDIUM  & MEDIUM & The importance of GITHUB as a whole, makes it a desirable target.\\\hline
Cisco ASA / VPN &  HIGH&  HIGH  & CRITICAL  & The Cisco switch has a known vulnerability, it sits on-prem and this could be exploted for confidential information. \\\hline
Cosmos DB       &  LOW & HIGH   & MEDIUM & This database contains sensitive information sent from device.         \\\hline
Azure Functions &  LOW & HIGH   & MEDIUM & Hals the upload of customer data and useage of custom function for HL7v2 translation.        \\\hline
Slack           &  MEDIUM & LOW  & LOW & Should slack be compromised, it is merely an internal communcations channel. No information about sensitive company data being shared using Slack.\\\hline
Google Cloud    &  LOW & LOW & LOW & Only historic use of this service, not noted as critical.      \\\hline
On-prem AD      &  HIGH & HIGH  & CRITICAL & The likelyhood could be easily reduced by removing known firewall vulnerabilities.\\\hline
Mediscan tools  &  HIGH & LOW   & MEDIUM   & Mediscan devices are in the hands of users. This increases the chances of exploits being exposed. \\\hline
Site-to-site VPN&  LOW  & HIGH & MEDIUM  & This is traffic on-prem to azure, if this is compromised, sensitive data can be accessed.\\
\hline
\end{tabular}
\caption{Summary of current providers and sovereignty-related dependency severity. }
\label{tab:risk table}
\end{table}

Likelyhood can then further be broken down into questions like 
\begin{itemize}
    \item Likelihood of legal or jusrisdictional issues
    \item Likelihood of operational issues
    \item Likelihood of vulnerabilities regarding network security
    \item Likelihood of policy shifting 
\end{itemize}

Given the overall dependency on software and services foreign to the EU, being mainly
US based, and given current political climates. It becomes increasingly difficult to 
narrow down the probablity that foreign governments, might try to leverage services
such as these, as political tools.


%brug en blød risk/impact matrix til den generelle fremgang,
%herefter link til https://owasp.org/www-community/OWASP_Risk_Rating_Methodology
%og giv et eksempel på hvordan det kunne se ud der fra, ikke nødvendigt at bruge billeder
%nævn at fokuset på rapporten er ikke risk management, men digital sovereignty

\subsection{Physical Vulnerabilities}
\subsubsection{Viby site}
The Viby office, inherited from HSM, which implemented rather modern 
security practices. They rely mainly on Azure as their infrastructre 
environment, thus, does not indicate any significant physical security risk.
However, since physical description of their offices has been supplied, it can only
be assumed that they have implemened a minimum of physical office security. This
includes thing like:
\begin{itemize}
    \item No physical servers
    \item light network equipment, routers/firewall
    \item secured doors, with keycards
    \item auto-lock screensavers
    \item CCTV of building
\end{itemize} 

It can only be assumed, due to the size and number of employees at HSM pre-merger, 
that the office was located in a shared-office building. This also points at things
they most certainly didn't implement

\begin{itemize}
    \item Biometric access
    \item Air-gapped security zones
    \item Cameras inside the offices
    \item Monitoring desk environments
    \item audit trails
\end{itemize}
    
These are things that more likely relate to ISO 27001\cite{ISO27001}, style physical access control,
which, due to their environments and workflows, most likely didn't require.

\subsubsection{Lystrup site}
The Lystrup manufacturing site, inherited from Mediscan, with dedicated engineering and
manufacturing environment, though it has been updated and improved over time. It still 
exposes several weakness, which has arisen from aging equipment.
\begin{itemize}
    \item Basic server room with no biometric access, meaning anyone with a keycard
    can enter the room.
    \item Basic UPS with a ambient cooling and a mix of new and old equipment. No temperature
    monitoring of server equipment, impacts lifespan and reliability.
    \item Cisco ASA firewall, unpatched since 2020. This particular model of switch is 
    vulnerable to a remote attack, which enables an attacker to gain memory content\cite{ASAfirewall}.
    \item legacy on-prem AD with no MFA, can be exploited to gain unrestricted access
    \item Physical access also allow the physical removal of decades of documentation, which
    could lead to IP theft.
\end{itemize}


\subsection{How Identified Risks Impacts LHT's Digital Soverignty }
The risks highlighted shows that the current landscape of LHT's sovereignty, is primarily
dictated by foreign, non-EU cload platforms, which includes a wide span of roles and services.
From the development environments, to azure AD and data processing, because these are US based
companies, they have to comply with legislation, such as the CLOUD act\cite{CLOUDACT} 
and FISA\_702\cite{FISA702}, this means that LHT does not fully control who gets access to
data. And that this data can be accessed and requested by foreign governments.\\

In addition to this, many of these non-EU services has a high degree of vender lock-in,
particularly, in this azure, that contains servers for both authentication, identity management, 
data ingestion and processing. This severely hampers the autonomy of LHT and its ability
to restructure the organization in the future.\\

The current physical vulnerabilities of the company, which is primarily found in the engineering
and manufacturing side of the organization, is affected by weak physical access controls, 
aging or degrading hardware and legacy software dependancies. This contributes an overall
security risk. 


%nævn det med cisco ASA FIREPOWER VULNERABILITY

%nævn AD forest og at den er federated

%nævn manufacturig systems med static IP 