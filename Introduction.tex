\section{Introduction}

%context
This report addresses the current and ongoing challenges that is present within Lifeline Health Technologies ApS 
due to the merger between Mediscan ApS, an established manufacturor of ultrasound electronics equipment 
for the health industry. And HealthSync Mobile ApS (HSM), a younger, digital health oriented company.
The two former companies originates from two different times, with differing missions and complex implementatitons.


%why this report exists

The complexity created by thtis merger and associated difficulties, due to differing it-landscapes, 
has prompted the executive team, to commission this report to highlight, identify and analyse the requirements and 
dependencies, related to digital Sovereignty with a focus on non-EU providers. The purpose of this report therefore
becomes to identify risks within the newly formed, yet disjointed, It invironment which LHT has inherited, with
its included assets, and how dependencies has could cause problems regarding regulatory compliance, operation stability
and long-term autonomy from non-EU actors.\\

%scope
The scope of this report covers the broader themes of digital Sovereignty, including identity and access management, data-storage, 
infrastructre and process mismatches, aswell as the risk associated with a futrure migration along with a focus on 
jurisdictional factors that arises when handling sensitive data or other regulated information. 

%methodology

%report structure
The overall structure of this report is, an overview of the current technology and service landscape of LHT, and a accompanying 
dependancy analysis. These two tie into the risk assessment of the current status. This is followed by an evaluation of possible
EU-alternatives to implemented services and comparison of the current implementatitoon. Lastly a recommended migration Strategy is 
proposed. 




% \begin{itemize}
%     \item Current technology and service landscape
%     \item Dependency Analysis
%     \item Risk assessment
%     \item An Evaluation of EU-alternatives
%     \item Comparison of current and proposed EU-friendly solutions
%     \item Recommended Migration Strategy
% \end{itemize}


