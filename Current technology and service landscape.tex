\section{Current Technology and Service Landscape}
As a result of the merger between Mediscan Aps and HealthSync Mobile Aps, the it environment, inherited to LHT, 
is comprised to two very different digital landscapes. These difference goes beyond mere technocal platforms, 
such as tool chosen for various tasks, but also divering filosophies. This leaves LHT to operation in a Hybrid environment,
where some parts of the company use incompatible identity systems, inconsistent governance practices and service providers
that might not be a perfect fit for EU regulations. 

\subsection{Mediscan technology landscape }
Mediscan originates from traditional engineering practies, around product development, production, calibration and
quality asurance. These include
\begin{itemize}
    \item On-prem EU-local infrastructre, there the whole it system is segmented into various sub networks, for engineering, 
    manufacturing and servers. 
    \item Local AD controllers, running on older operating systems, with weak overall governance
    \item Weak network security regarding hardware, older switches with no updates
    \item No MFA and relying on password update rotations
\end{itemize}

The infrastructre inherited from Mediscan, shows strong prefence to local storage and authentication along with a 
typical product first approach, where cynersecurity is seen as a hinderance. 

\subsection{HealthSync Mobile's technology landscape}
HSM oprates a rather modern techonology environment, based in the new cloud computing era. With a heavy reliance
on popular service providers such as the Microsoft Azure suite, Google cloud and other Saas products. 
These include:
\begin{itemize}
    \item An extensize implementation of Azure services such as: AD, kubernetes, Cosmos DB, PostgreSQL, Key Vault, Api 
    management, blob storage and monitor + application insights.
    \item Heavy reliance on CI/CD pipelines and API updates, from non-EU cloud hosted services. 
    \item JSON/WIRE-formatted data.
    \item Enforced MFA, and OAuth2 + OpenID for access to apps
\end{itemize}

HSM's modern Technology stack, which histroically has granted them the capability for rapid expansion and growth
comes at the cost of strong ties to non-EU governed cloud services. Which could become subject to outside EU regulation.



\subsection{The hybrid environment post merger}
The post-merger leaves LHT with a partially integrated and partially improved hybrid network, currently consisting of:
\begin{itemize}
    \item The legacy on-prem VLAN of Mediscan's solution for manufacturing and engineering
    \item HSM's Azure environment and dependencies
    \item Site-to-site VPN between the two company environments

\end{itemize}
Inconsistent and overlapping infrastructures creates uncertain flow between different parts of
the company. This poses unknown and unmitigated risks, which could be leveraged as attack vectors 
and entry points.  

% Additional complexities regarding legislation
% both inside and outside of the EU.


\subsection{Identity and Access management} %D3.1 D3.2'
An important integration issue is the overlapping domains, which results in some users need 
separate accounts to access multiple services. This leads to multiple services for the same purpose.
This also introduces conflicting policies.

\begin{itemize}
    \item Some users having multiple accounts, both on-prem and cloud.
    \item Azure AD policy, excludes the use of legacy hardware primarily found in Mediscan
    \item Inconsistent access rollout to internal file services such as Sharepoint.
\end{itemize}

Unnecessarily complicated access controls, which would require significant changes to the operation
of the company infrastructre.


% for risk assessment
% And because no company managed password manager
% has been implemented, this increases d






% cloud servises
% SaaS platform
% networking and security 
% hardware vendors talk about the current switches and vulnerabilities
