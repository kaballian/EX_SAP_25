\section{Comparison of current and proposed solutions}
The Suitablility of these EU derived alternatives will here be evaulated against the already 
implemented solution, in terms of technical compatability, which reflects how well this
solution fits into the current LHT environment. And vendor maturity, which assesses the stability 
of particular provider regarding support, regulation compliance and long-term viability.

\subsection{Technical Compatability}
Technical compatability assesses the difficulty of implementing the proposed solutions,
into the environment of LHT. Replacing all Azure services and with suitable replacements,
which offers equivalent services, in this case Univention for identity management and Scaleway
for hosting for cloud computte and storage, reduces the implementation effort, thus increases the chance 
of a smooth migration

Examples of this is the Active directory from Univention, which replaces Azure AD, with minimal
changes to already implemented authentication dependencies. Univention also provides
NUBUS with keycloak, which preserves the use of OAuth2 and OpenID for
customer usageage. While changing the underlying structure to a self-hosted environment,
located in the EU. 

Distributing the services across a number of different EU providers, increases the 
flexibility and should reduce the long term vendor lock-in. An example of this
is ScaleWay, which can do both cloud compute, data storage and processing.
But also host a GIT server and other required developer tooling, but with smaller
requirements for future changes. 




\subsection{Vendor Compatability}
Vendor maturity assesses the long-term reliablity, stability, regulation compliance and support.
It is imperative for LHT, that 


%SLET
Vendor maturity evaluates the long-term reliability, stability, governance model, 
and support capabilities of the proposed EU-based providers. Mature vendors reduce 
operational uncertainty, lower the risk of service disruption, and provide clearer 
guarantees regarding security, compliance, and roadmap stability.

In this regard, both Univention and Scaleway demonstrate strong maturity within the 
European technology landscape. Univention has a long-standing presence in the public sector and enterprise markets, with a proven track record of maintaining its identity platform and supporting large-scale Active Directory migrations. Its governance model is fully EU-based, and the long-term availability of UCS and Keycloak integrations is reinforced by active open-source communities and transparent development practices. This increases confidence in the sustainability of the solution and reduces the likelihood of unexpected product changes or jurisdictional risks.

Scaleway likewise represents a technically mature and operationally stable EU cloud 
provider, offering compute, storage, and networking services that align closely with 
established European regulatory frameworks. As an established IaaS provider hosting
 thousands of workloads, Scaleway maintains certifications and a clear commitment to 
 GDPR-aligned cloud governance. Their multi-service portfolio also contributes to vendor
  stability, ensuring that LHT is not dependent on a niche or single-product supplier.

Overall, the proposed alternatives come from vendors with sufficient operational
 maturity and market presence to act as long-term strategic components in LHT's architecture. 
 By selecting EU-governed providers with transparent roadmaps and strong support ecosystems, 
 LHT reduces dependency risk, improves strategic autonomy, and avoids the extraterritorial 
 exposure associated with non-EU cloud providers.
