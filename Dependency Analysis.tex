\section{Dependency Analysis}


\subsection{Purpose}
The purpose of a dependency analysis is to map where curial dependencies lie. This makes easier to prevent
certain types of attacks, such as those those aimed at particular spots in a company, be it a supply chain or something else.
It is therefore imperative that service providers are vetted, to prevent vulnerabilities, which could undermine
the efforts to ensure the safety of the company.

\subsection{Dependency mapping}
From the merger, HLS has inherited a multitude of dependencies. In the section below, the different
types are listed:

\subsubsection{A. Cloud infrastructre Dependencies}
\begin{itemize}
    \item Azure \begin{itemize}
        \item Azure DevOps
        \item Azure functions, to deploy custom script for legacy HL7 v2 conversion.
    \end{itemize}
    \item Google Cloud services
    
\end{itemize}
\subsubsection{B. Identity \& Access management Dependencies}
\begin{itemize}
    \item Azure \begin{itemize}
        \item Azure AD 
        \item Azure AD connect 
        \item Azure B2C
    \end{itemize}
    \item On-prem AD
\end{itemize}
\subsubsection{C. Application \& SaaS Dependencies}
\begin{itemize}
    \item GitHub
    \item Microsoft Teams
    \item Slack
\end{itemize}
\subsubsection{D. Security \& Monitoring Dependencies}
\begin{itemize}
    \item Cisco \begin{itemize}
        \item Cisco ASA firewall, last firmware update from 2020 - legacy
        \item multiple managed Cisco ethernet switches
        \item VPN - Cisco any connect.
        \item VPN - Site-to-Site 
    \end{itemize}
    
\end{itemize}

\subsubsection{E. Data processing \& Clinical workflow Dependencies}
\begin{itemize}
    \item Azure \begin{itemize}
        \item Azure Blob storage
        \item Azure CosmosDB
    \end{itemize}
    \item Custom solutions \begin{itemize}
        \item C++ device-configuration tool(local)
        \item HL7v2 to JSON translation scripts running in Azure functions
    \end{itemize}
    \item Microsoft \begin{itemize}
        \item OneDrive
        \item SharePoint
    \end{itemize}
    \item DICOM handling
\end{itemize}

\subsubsection{F. Development and Tooling Dependencies}
\begin{itemize}
    \item GitHub 
    \item Terraform
    
\end{itemize}


\subsection{Current providers and dependency severity}  



\begin{table}[htb]
\centering
\renewcommand{\arraystretch}{1.2}
\begin{tabular}{|l|l|l|c|c|c|}
\hline
\textbf{Provider / Service} &
\textbf{Jurisdiction} &
\textbf{Area} &
\textbf{Criticality} &
\textbf{Sov. Risk} &
\textbf{Lock-in}\\
\hline

Microsoft Azure & US & Cloud compute/storage    & High & High       & High      \\\hline
Azure AD / B2C  & US & Identity management      & High & High       & High      \\\hline
Microsoft 365   & US & Collaboration \& files   & High & High       & Medium     \\\hline
GitHub          & US & Source code / CI         & High & High       & Low      \\\hline
Cisco ASA / VPN & US & Network perimeter        & High & Medium     & Medium      \\\hline
Cosmos DB       & US & Clinical metadata        & High & High       & Medium      \\\hline
Azure Functions & US & HL7 ingestion            & High & High       & High      \\\hline
Slack           & US & Communication (legacy)   & Medium & Medium   & Low      \\\hline
Google Cloud    & US & Analytics / telemetry    & Medium & High     & Low      \\\hline
On-prem AD      & EU & Manufacturing IAM        & High & Low        & High      \\\hline
Mediscan tools  & EU & Device config            & Medium & Low      & High      \\\hline
AnyConnect VPN  & US & Remote access            & High & Medium     & Medium      \\
\hline
\end{tabular}
\caption{Summary of current providers and sovereignty-related dependency severity.}
\label{tab:dependency}
\end{table}

%format the tabler later, use other names for Area description - remember lock in

In table: \ref{tab:dependency} the columns, \textit{Criticality} refers to the company's current 
depency on the current implementation. \textit{Sovereignty Risk} explains how much control the company loses
over its assets, as they are hosted by these provices, and lastly \textit{Lock-in} aims to indicate
how difficult it would be to replace the service, if it was necessary.
%omformuler












